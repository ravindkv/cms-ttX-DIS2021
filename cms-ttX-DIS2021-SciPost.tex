% =========================================================================
% SciPost LaTeX template
% Version 1e (2017-10-31)
%
% Submissions to SciPost Journals should make use of this template.
%
% INSTRUCTIONS: simply look for the `TODO:' tokens and adapt your file.
%
% - please enable line numbers (package: lineno)
% - you should run LaTeX twice in order for the line numbers to appear
% =========================================================================


% TODO: uncomment ONE of the class declarations below
% If you are submitting a paper to SciPost Physics: uncomment next line
%\documentclass[Physsubmission, Phys]{SciPost}
% If you are submitting a paper to SciPost Physics Lecture Notes: uncomment next line
%\documentclass[submission, LectureNotes]{SciPost}
% If you are submitting a paper to SciPost Physics Proceedings: uncomment next line
\documentclass[submission, Proceedings]{SciPost}

%-------------------------------------------------
% CERN-LHC-CMS specific styles
%-------------------------------------------------
\usepackage{lineno}
\linenumbers

\newcommand{\fbinv}{\ensuremath{{\text{fb}^{-1}}}{}}
\newcommand{\GeV}{\ensuremath{{\text{GeV}}}}
\newcommand{\TeV}{\ensuremath{{\text{TeV}}}}
\newcommand{\ptmiss}{\ensuremath{{p_\text{T}^{\text{miss}}}}}
\newcommand{\pt}{\ensuremath{{p_\text{T}}}}

\newcommand{\Pe}{\ensuremath{{e}}}
\newcommand{\PQt}{\ensuremath{{t}}}
\newcommand{\PAQt}{\ensuremath{{\bar{t}}}}
\newcommand{\PQb}{\ensuremath{{b}}}
\newcommand{\PAQb}{\ensuremath{{\bar{b}}}}
\newcommand{\PQc}{\ensuremath{{c}}}
\newcommand{\PAQc}{\ensuremath{{\bar{c}}}}
\newcommand{\ttbar}{{\PQt\PAQt}}
\newcommand{\PGg}{\ensuremath{{\upgamma}}}
\newcommand{\PZ}{\ensuremath{{Z}}}
\newcommand{\PW}{\ensuremath{{W}}}
\newcommand{\Pepm}{\ensuremath{{e^\pm}}}
\newcommand{\PGmmp}{\ensuremath{{\mu^\pm}}}


\newcommand{\PQtau}{\ensuremath{{\tau}}}
\newcommand{\Pmu}{\ensuremath{{\mu}}}
\newcommand{\pzvar}{\ensuremath{p_\text{Z}(\Pepm, \PGmmp, j)}}
\newcommand{\invmassvar}{\ensuremath{m(\Pepm, \PGmmp,j)}}
\newcommand{\transmassvar}{\ensuremath{m_{\text{T}}(\Pepm, \PGmmp,j,\ptmiss)}}
\newcommand{\deltaPhiVar}{\ensuremath{\Delta\varphi(\Pepm, \PGmmp)}}
\newcommand{\ljets}{\ensuremath{\ell + \text{jets}}}
\newcommand{\dilep}{\ensuremath{2\ell + \text{jets}}}
\newcommand{\ttjets}{\ensuremath{\ttbar + \text{jets}}}
\newcommand{\ttgamma}{\ensuremath{\ttbar\gamma}}
\newcommand{\ttcc}{{\ttbar\PQc\PAQc}}
\newcommand{\ttcL}{{\ttbar\PQc\text{L}}}
\newcommand{\ttbb}{{\ttbar\PQb\PAQb}}
\newcommand{\ttbL}{{\ttbar\PQb\text{L}}}
\newcommand{\ttLF}{{\ttbar\text{LL}}}
\newcommand{\ttjj}{{\ttbar\text{jj}}}
\newcommand{\mujets}{\ensuremath{\mu + \text{jets}}}
\newcommand{\ejets}{\ensuremath{\Pe + \text{jets}}}

\binoppenalty=10000
\relpenalty=10000

\hypersetup{
    colorlinks,
    linkcolor={red!50!black},
    citecolor={blue!50!black},
    urlcolor={blue!80!black}
}

\usepackage[bitstream-charter]{mathdesign}
\urlstyle{sf}

% Fix \cal and \mathcal characters look (so it's not the same as \mathscr)
\DeclareSymbolFont{usualmathcal}{OMS}{cmsy}{m}{n}
\DeclareSymbolFontAlphabet{\mathcal}{usualmathcal}

\begin{document}

% TODO: write your article's title here.
% The article title is centered, Large boldface, and should fit in two lines
\begin{center}{\Large \textbf{
Recent measurements of top quark production cross sections in CMS\\
}}\end{center}

% TODO: write the author list here. Use initials + surname format.
% Separate subsequent authors by a comma, omit comma at the end of the list.
% Mark the corresponding author with a superscript *.
\begin{center}
Ravindra K Verma on behalf of the CMS Collaboration\textsuperscript{$\star$} 
\end{center}

% TODO: write all affiliations here.
% Format: institute, city, country
\begin{center}
Florida Institute of Technology, USA 
\\
% TODO: provide email address of corresponding author
*rverma@cern.ch 
\end{center}

\begin{center}
\today
\end{center}

% For convenience during refereeing (optional),
% you can turn on line numbers by uncommenting the next line:
%\linenumbers
% You should run LaTeX twice in order for the line numbers to appear.

\definecolor{palegray}{gray}{0.95}
\begin{center}
\colorbox{palegray}{
  \begin{tabular}{rr}
  \begin{minipage}{0.1\textwidth}
    \includegraphics[width=22mm]{Logo-DIS2021.png}
  \end{minipage}
  &
  \begin{minipage}{0.75\textwidth}
    \begin{center}
    {\it Proceedings for the XXVIII International Workshop\\ on Deep-Inelastic Scattering and
Related Subjects,}\\
    {\it Stony Brook University, New York, USA, 12-16 April 2021} \\
    \doi{10.21468/SciPostPhysProc.?}\\
    \end{center}
  \end{minipage}
\end{tabular}
}
\end{center}

\section*{Abstract}
{\bf
Measurements of the cross section for the production of the single top with a \PW boson, top quark pair
with and without additional quarks are presented at the different center of mass energies with different
integrated luminosity recorded by the CMS experiment during the years 2016-18 of data-taking. The 
measurement is performed in semi and dileptonic decay modes of the top quark. A few of them 
are performed for the first time. The measured and the predicted value of the inclusive cross 
section are in agreement within the uncertainties. The differential and double differential
measurement has slight over-prediction in a certain kinematic region due to the mismodeling of 
the event generator.
}

\section{Introduction}
\label{sec:intro}
The top quark is one of the most interesting particles present in the Standard Model (SM) of
particle physics as it is postulated to interact with unknown particles beyond the SM
theories. Precise measurement of the properties of the top quark helps in the improvement of 
search sensitivity and test of perturbative Quantum Chromodynamics. Differential cross section 
measurement is used to test fixed-order predictions and extract QCD parameters. The \ttbar~ 
production cross section is dominant at LHC and serves as the background for many new physics searches.

There have been many earlier measurements at 7, 8, and 13 TeV center of mass energy 
as shown in Figure~\ref{fig:xss}. Among them, the \ttbar~ production dominates.
In this proceeding, we summarise the recent cross section measurement in the CMS experiment at the
LHC. For a more detailed description of each of the measurements, the reader is encouraged to look
at the corresponding paper. Here we summarise the \ttbar~ measrements in Section~\ref{sec:tt},
\PQt\PW~ in Section~\ref{sec:tW}, and $\PQt\PQt$X in Section~\ref{sec:ttX} where X stands for $\gamma$, $c\bar{c}$, and $b\bar{b}$.

\begin{figure}[htb!]
\centering
\includegraphics[width=1.0\linewidth]{Figures/SigmaNew_v8.pdf}
\caption{Top quark prodution cross section at various center of mass energies measured in the
    CMS experiment. The measurement is performed for the production of single \PQt, \ttbar~, \ttbar~ + jets, \PQt + X, \ttbar~+ X, \ttbar\ttbar, etc. Among these, a few of them are new at 13 TeV whereas a
    few of them are still being studied. There is a good agreement between the predicted and
    measured values within the uncertainties.}
\label{fig:xss}
\end{figure}

%================
% tt
%================
\section{\ttbar~ production}
\label{sec:tt}
Due to the higher cross section, the \ttbar~ production process is extensively studied at the LHC.
Depending on the subsequent decays of the two \PQt~ quarks, the final states consist of either all
jets, \ljets, or \dilep. All jet final states have more statistics but more multijet background whereas the \dilep~ have fewer statistics but small background. The \ljets final states fall in 
between the two. However, the measurement of cross sections from all these final states is needed
in order to study various physics beyond the SM. In this proceedings, we present the measurement 
from the \ljets and \dilep~ final states for \ttbar~ production.


\subsection{Inclusive and differential in the \ljets final states}
The measurement of the cross section is performed using 137 \fbinv integrated luminosity in the 
\ejets~ and \mujets~ final states~\cite{CMS-PAS-TOP-20-001}. In order to increase the sensitivity, 
the events are further divided into boosted and resolved categories based on the kinematics of the
decay products of the \ttbar. The final cross section is extracted by a simultaneous fit combining 
events from both final states and all categories. Various distributions such as the transverse 
momentum of the \PQt~ quark, invariant mass of \ttbar, etc are used to measure the differential and 
double differential cross-sections at parton and particle levels. A neural network is exploited in the 
reconstruction of variables from boosted \PQt~ quark. A $\chi^2$ test is performed to compare the 
measurements with several predictions. The dominant source of systematic uncertainty comes from 
jet energy correction.

The meausred value of the inclusive cross section $791\pm 25$ pb 
is in agreement with the corresponding predicted value of $832\pm 46$ pb. One of the most 
striking features of this measurment is that the measured cross section is more precise 
(3.2\% uncertainty) as compared to the predicted value (5.5\% uncertainty). The measured and 
predicted differential and double differential cross sections as functions of various variables are 
in agreement within the uncertainties for most of the variables. However, there is a slight 
discrepancy in the double differential measurement for higher \pt~ of the hadronically decaying \PQt~
quark in the range $ 0 < \pt (\ttbar) < 120 \GeV$). A similar over-prediction is 
also observed from the ATLAS experiment~\cite{ATLAS:2020ccu} which signals a possible mismodeling 
of the Monte-Carlo event generator used for the simulation of predicted events. 

\subsection{Inclusive in the \dilep~ final states at 5.02 TeV}
As shown in Figure~\ref{fig:xss}, all measurements are performed at 7, 8, and 13 TeV center of
mass energies. This is the first measurement at 5.02 TeV which provides another test for the 
SM at lower energy~\cite{CMS-PAS-TOP-20-004}. The measurement is performed using 0.304\fbinv 
integrated luminosity in the \ejets~ and \mujets~ final states. The cross section is extracted 
by performing the fit on total number of events after applying all selections. The dominant source 
of systematic uncertainty comes from the jet energy correction. The predicted cross section at the
next-to-leading order (NLO) in QCD and observed value are 
$66.8^{+1.9}_{-2.3}\text{(scale)}\pm 1.7\text{(PDF)}^{+1.4}_{-1.3}(\alpha_S)$ pb and 
$60.3 \pm 5.0 \text{(stat)} \pm 2.8 \text{(syst)} \pm 0.9 \text{(lumi)}$ pb, respectively. They
agree within the uncertainties. 

\subsection{Inclusive in the \dilep~ final states including $\PQtau$}
This is the first measurement involving $\tau$ lepton~\cite{CMS:2019snc} and provides another 
way of checking lepton flavor universality violation. With the third generation of lepton and quarks,
it is sensitive to beyond SM contributions such as the reconstruction of charged Higgs boson. In 
this analysis, one of the lepton from \dilep~ final states is required to be hadronically decaying 
\PQtau~ and the other one is either an electron or muon. The masurement is performed at 13 TeV 
using 35.9\fbinv integrated luminosity. The cross section is extracted using the profile 
likelihood method based on the transverse mass of the lepton and missing transverse energy.The  
QCD multijet background is estimated from the data. The main sources of systematic uncertainty are 
from $\tau_h$ identification and misidentification. The measured cross section combining both channels 
is  
$\sigma_\ttbar(\ell\tau_h) = 781 \pm 7 \text{(stat)} \pm 62 \text{(syst)} \pm 20 \text{(lumi)}$ pb
which is in agreement with the corresponding predicted value. The ratio of this with the same 
flavor cross section 
$R_{\ell\tau_h/\ell\ell} = 0.973 \pm 0.009 \text{(stat)} \pm 0.066 \text{(syst)}$ is close to 1 
within the uncertainties. Hence the lepton flavor universality violation is not observed.

%================
% tW
%================
\section{\PQt\PW~ production} 
\label{sec:tW}
The \PQt\PW~ production process is one of the sub-dominants in terms of the total cross section as 
shown in Figure~\ref{fig:xss}. It is also sensitive to the relevant CKM matrix element. 
Any deviation from the predicted value may be indicative of physics beyond the SM. Similar to the 
\ttbar~ measurement, we present the recent studies from the \ljets and \dilep~ final states.

\subsection{Inclusive in the \ljets final states}
The measurement is performed in the \ejets~ and \mujets~ final states with 35.9\fbinv integrated
luminosity~\cite{CMS-PAS-TOP-20-002}. An event-level discriminant based on Boosted Decision Tree 
is used to extract the cross section. The events are divided into different signal and control 
regions based on the number of jets and b-tagged jets. A simultaneous fit is performed on the
discriminant combining 3 categories and 2 final states. The dominant source of systematic 
uncertainty comes from the jet energy correction. The predicted (at NLO) and 
measured cross sections are $79.5^{+1.9+2.0}_{-1.8-1.4}$ pb and 
$89 \pm 4 (\text{stat}) \pm 12 (\text{syst})$ pb, respectively. They are in agreement within the
uncertainties.

\subsection{Differential in the \dilep~ final states}
The differential cross section is measured with 35.9\fbinv integrated luminosity in the \dilep~ final
states with different lepton flavor (electron or muon) as a function of the six 
variables~\cite{CMS-PAS-TOP-19-003}: \pt~ of the leading lepton, \pt~ of the jet, angular difference
\deltaPhiVar, longitudinal momentum \pzvar, invariant mass \invmassvar, and transverse mass 
\transmassvar. Signal extraction is performed by subtracting backgrounds from data. The jet energy 
correction uncertainties are the dominant ones. Predicted and measured cross sections are in 
agreement within the uncertainties across different bins of all variables.

%================
% ttX
%================
\section{\ttgamma, \ttcc, \ttbb~productions}
\label{sec:ttX}
Although the cross section of these processes are smaller as shown in Figure~\ref{fig:xss},
they are useful in studying rare phenomenon with in the SM and new physics beyond it, for example,
The \ttgamma~ measurement allows in constraining the $\PQt\gamma$ electroweak coupling and \ttcc~ or
\ttbb~ provide a useful test of NLO QCD calculations. 

\subsection{Inclusive and differential \ttgamma~ in $\ell$ + jets final states}
This is the first \ttgamma~ cross section measurement at 13 TeV in the CMS experiment using 137\fbinv~
integrated luminosity~\cite{CMS-PAS-TOP-18-010}. The measurement is performed in the \ejets~ and 
\mujets~ final states with one photon. Photon is classified based on matched generator parton in 
the genuine, nonprompt, misidentified, and multijet photon cateforied. Different phase space based 
on object selections and kinematic cuts are exploited to improve the precision. QCD multijet and 
electroweak backgrounds are measured from data. A simultaneous fit combining all event categories is 
performed to extract the cross section. The dominant uncertainties in the cross section come from 
$\PW\gamma$ normalization and misidentified $\gamma$ estimation. The measured value of the inclusive
cross section in  the fiducial phase space is $800 \pm 46 (\text{syst}) \pm 7 (\text{stat})$ fb.
The ratio of the measured and predicted (at NLO) cross section is $1.034^{+0.060}_{-0.058}$,
that is, they agree within the uncertainties. There is also good agreement for the differential 
cross section in most bins of \pt~ and $\eta$ of the photon. Though there is a slight over-prediction
in the bins where statistical precision is low. 

\subsection{Inclusive \ttcc~ cross section in $2\ell$ + jets final states}
Due to the availability of \PQc-jet taggers at 13 TeV, there has been an improvement in the 
sensitivity in the measurement involving \PQc~ jet in the final state. For the first time \ttcc~ 
cross section measurement is performed in the CMS~\cite{CMS:2020qvt}. The analysis is performed in 
the \dilep~ final states with the same flavor lepton (\Pe, \Pmu, or \PQtau) with 41.5\fbinv~ integrated 
luminosity at the center-of-mass energy of 13 TeV. A neural network is trained to distinguish 
between top quark pair events with additional jets. Event level neural network predicts output 
probabilities for five output classes $P$($\ttcc$), $P$($\ttcL$), $P$($\ttbb$), $P$($\ttbL$), and $P$($\ttLF$). Two variables are derived based on these
\begin{linenomath}
\begin{equation}
  \begin{aligned}
\Delta_{\PQb}^{\PQc} &= \frac{P(\ttcc) }{P(\ttcc) + P(\ttbb)},\\
\Delta_{\text{L}}^{\PQc} &=  \frac{P(\ttcc)}{P(\ttcc) + P(\ttLF)}.
\label{eq:Dbcdiscriminator}
  \end{aligned}
\end{equation}
\end{linenomath}
A 1-d histogram is constructed from the 16 bins of the 2-d plane of these two variables
\begin{equation}
\Delta_{\text{L}}^{\PQc} \otimes \Delta_{\PQb}^{\PQc} :  [0,0.55,0.65,0.85,1.0] \otimes [0,0.35,0.5,0.6,1.0].
\end{equation}
The \ttcc, \ttbb, \ttLF~cross sections are simultaneously extracted by fitting the neural newtwork
outputs from simulation and observation. The dominant source of systematic uncertainty comes 
from the jet energy correction and c-tagging calibration. The $\ttcc$ cross section in the
full phase space is measured to be $7.43\pm 1.07(\text{stat})\pm 0.95(\text{syst})$ pb. 
An overall agreement is observed between the measured and predicted value at the level of 
one to two standard deviations for the \ttcc, \ttbb, and \ttLF~ processes. 

\subsection{Inclusive \ttbb~ cross sections in \ljets~ and \dilep~ final states}
The measurement is performed at 13 TeV with 35.6\fbinv~ integrated luminosity~\cite{CMS:2020grm}. 
The cross section is separately extracted for both final states for \ttbar~ and \ttjj~ and their ratio
in the visible and full phase space. The fit is performed on the b-tagging discriminant value of 
the two jets. A 1-d histogram is constructed from the 10×10(20×20) bins of the 2-d plane of these 
two variables for \ljets (\dilep) final states. Theoretical uncertainties from the final state
radiation and madevent parton-shower matching are dominant. The corresponding measured cross
section for \ttbb~ is $0.040 \pm 0.002 (\text{stat}) \pm 0.005 (\text{syst})$ pb and 
$0.62 \pm 0.03 (\text{stat}) \pm 0.07 (\text{syst})$ pb for \ljets and \dilep~ final states, 
respectively. These are consistent within the uncertainties, with the SM prediction obtained using 
a matrix element calculation at NLO order in QCD matched to a parton shower.
 

\section{Conclusion}
In this proceedings, a summary of the inclusive, differential, and double differential cross section
measured in the CMS experiment is presented for different final states. The measured value of 
\ttbar~ cross section is more precise as compared to the predicted one. A slight over-prediction
is seen in the differential \ttgamma~ due to low statistics and double differential \ttbar~ cross 
section due to mismodeling of the \pt~ spectrum of the \ttbar~ system. The \ttbar~ mesurement at
5.02 TeV provides another way of testing the consistency of SM prediction with the observation
at lower energy. For the first time, the final states involving \PQtau~ lepton are analyzed for the
\ttbar~ process. Other measured cross sections for the \PQt\PW, \ttgamma, \ttcc~ and \ttbb~ 
production processes are in agreement with SM prediction within the systematic and statistical 
uncertainties.


\section*{Acknowledgements}
We congratulate our colleagues in the CERN accelerator departments for the excellent performance of the LHC
and thank the technical and administrative staffs at CERN and at other CMS institutes for their contributions
to the success of the CMS effort. In addition, we gratefully acknowledge the computing centres and personnel
of the Worldwide LHC Computing Grid for delivering so effectively the computing infrastructure essential to
our analyses. Finally, we acknowledge the enduring support for the construction and operation of the LHC and
the CMS detector provided by the following funding agencies: BMBWF and FWF (Austria); FNRS and FWO (Belgium);
CNPq, CAPES, FAPERJ, FAPERGS, and FAPESP (Brazil); MES (Bulgaria); CERN; CAS, MoST, and NSFC (China);
COLCIENCIAS (Colombia); MSES and CSF (Croatia); RPF (Cyprus); SENESCYT (Ecuador); MoER, ERC IUT, PUT and ERDF
(Estonia); Academy of Finland, MEC, and HIP (Finland); CEA and CNRS/IN2P3 (France); BMBF, DFG, and HGF
(Germany); GSRT (Greece); NKFIA (Hungary); DAE and DST (India); IPM (Iran); SFI (Ireland); INFN (Italy); MSIP
and NRF (Republic of Korea); MES (Latvia); LAS (Lithuania); MOE and UM (Malaysia); BUAP, CINVESTAV, CONACYT,
LNS, SEP, and UASLP-FAI (Mexico); MOS (Montenegro); MBIE (New Zealand); PAEC (Pakistan); MSHE and NSC
(Poland); FCT (Portugal); JINR (Dubna); MON, RosAtom, RAS, RFBR, and NRC KI (Russia); MESTD (Serbia); SEIDI,
CPAN, PCTI, and FEDER (Spain); MOSTR (Sri Lanka); Swiss Funding Agencies (Switzerland); MST (Taipei);
ThEPCenter, IPST, STAR, and NSTDA (Thailand); TUBITAK and TAEK (Turkey); NASU and SFFR (Ukraine); STFC (United
Kingdom); DOE and NSF (USA).

%\begin{verbatim}
\bibliographystyle{unsrt}
\bibliography{cms-ttX-DIS2021-SciPost.bib}
%\end{verbatim}

\nolinenumbers
\end{document}
