 %================
 % ttX
 %================
 Although the cross sections of these processes are smaller as shown in Figure~\ref{fig:xss},
 they are useful in studying rare phenomena within the SM and new physics beyond it, for example, 
 the \ttgamma measurement allows in constraining the $\PQt\gamma$ electroweak coupling and $\ttcc$ or
 \ttbb provide a useful test of NLO QCD calculations. 

 \subsection{Inclusive and differential \texorpdfstring{\ttgamma}{ttgamma} cross section measurements in \texorpdfstring{\ljets}{ljets} final states}
 This is the first \ttgamma cross section measurement at 13 \TeV by the CMS experiment using 137\fbinv
 integrated luminosity~\cite{CMS-PAS-TOP-18-010}. The measurement is performed in the \ejets and 
 \mujets final states with one photon. Photons are classified based on matched generator parton in 
 the genuine, nonprompt, misidentified, and multijet photon categories. Different phase spaces based 
 on object selections and kinematic cuts are exploited to improve the precision. QCD multijet and 
 electroweak backgrounds are measured from data. A simultaneous fit combining all event categories is 
 performed to extract the cross section. The dominant uncertainties in the cross section comes from 
 $\PW\gamma$ normalization and misidentified $\gamma$ estimation. The measured value of the inclusive
 cross section in  the fiducial phase space is $800 \pm 46 (\text{syst}) \pm 7 (\text{stat})$ fb.
 The ratio of the measured and predicted (at NLO) cross section is $1.034^{+0.060}_{-0.058}$,
 that is, they agree within the uncertainties. There is also good agreement for the differential 
 cross section in most bins of \pt and $\eta$ of the photon. Though there is a slight over-prediction
 in the bins where statistical precision is low. 

 \subsection{Inclusive \texorpdfstring{\ttcc}{ttcc} cross section measurement in \texorpdfstring{\dilep}{dilep} final states}
 Due to the availability of \PQc-jet taggers at 13 \TeV, there has been an improvement in the 
 sensitivity in the measurement involving \PQc jet in the final state. For the first time, a \ttcc 
 cross section measurement is performed  by the CMS Collaboration~\cite{CMS-PAS-TOP-20-003}. The 
 analysis is performed in the \dilep final states with the same flavor lepton (\Pe, \Pmu, or \PGt) 
 with 41.5\fbinv integrated luminosity at the center-of-mass energy of 13 \TeV. A neural network is 
 trained to distinguish between top quark pair events with additional jets. Event level neural 
 network predicts output probabilities for five output classes $P$($\ttcc$), $P$($\ttcL$), $P$($\ttbb$), $P$($\ttbL$), and $P$($\ttLF$). Two variables are derived based on these
 \begin{linenomath}
 \begin{equation}
   \begin{aligned}
 \Delta_{\PQb}^{\PQc} &= \frac{P(\ttcc) }{P(\ttcc) + P(\ttbb)},\\
 \Delta_{\text{L}}^{\PQc} &=  \frac{P(\ttcc)}{P(\ttcc) + P(\ttLF)}.
 \label{eq:Dbcdiscriminator}
   \end{aligned}
 \end{equation}
 \end{linenomath}
 A 1-d histogram is constructed from the 16 bins of the 2-d plane of these two variables
 \begin{equation}
 \Delta_{\text{L}}^{\PQc} \otimes \Delta_{\PQb}^{\PQc} :  [0,0.55,0.65,0.85,1.0] \otimes [0,0.35,0.5,0.6,1.0].
 \end{equation}
 The \ttcc, \ttbb, \ttLF cross sections are simultaneously extracted by fitting the neural network
 outputs from simulation and observation. The dominant source of systematic uncertainty comes 
 from the jet energy correction and c-tagging calibration. The $\ttcc$ cross section in the
 full phase space is measured to be $7.43\pm 1.07(\text{stat})\pm 0.95(\text{syst})$ pb. 
 An overall agreement is observed between the measured and predicted value at the level of 
 one to two standard deviations for the \ttcc, \ttbb, and \ttLF processes. 

 \subsection{Inclusive \texorpdfstring{\ttbb}{ttbb} cross section measurements in \texorpdfstring{\ljets}{ljets} and \texorpdfstring{\dilep}{dilep} final states}
 The measurement is performed at 13 \TeV with 35.6\fbinv integrated luminosity~\cite{CMS:2020grm}. 
 The cross section is separately extracted for both final states for \ttbar and \ttjj and their ratio
 in the visible and full phase space. The fit is performed on the b-tagging discriminant value of 
 the two jets. A 1-d histogram is constructed from the 10x10(20x20) bins of the 2-d plane of these 
 two variables for \ljets (\dilep) final states. Theoretical uncertainties from the final state
 radiation and madevent parton-shower matching are dominant. The corresponding measured cross
 section for \ttbb in the full phase space is $4.7 \pm 0.2 (\text{stat}) \pm 0.6 (\text{syst})$ pb 
 and $2.9 \pm 0.1 (\text{stat}) \pm 0.5 (\text{syst})$ pb for \ljets and \dilep final states, 
 respectively. These are consistent within the uncertainties, with the SM prediction obtained using 
 a matrix element calculation at NLO order in QCD matched to a parton shower.

 \section{Conclusion}
 In this proceedings, a summary of the inclusive, differential, and double differential cross section
 measured by the CMS experiment is presented for different final states. The measured value of 
 \ttbar cross section is more precise as compared to the predicted one. A slight over-prediction
 is seen in the differential \ttgamma due to low statistics and double differential \ttbar cross 
 section due to mismodeling of the \pt spectrum of the \ttbar system. The \ttbar measurement at
 5.02 \TeV provides another way of testing the consistency of SM prediction with the observation
 at lower energy. For the first time, the final states involving \PGt lepton are analyzed for the
 \ttbar process. Other measured cross sections for the \PQt\PW, \ttgamma, \ttcc and \ttbb 
 production processes are in agreement with SM prediction within the systematic and statistical 
 uncertainties.

