 %================
 % tW
 %================
 The \tW production process is one of the sub-dominants in terms of the total cross section as 
 shown in Figure~\ref{fig:xss}. It is also sensitive to the relevant CKM matrix element. 
 Any deviation from the predicted value may be indicative of physics beyond the SM. Similar to the 
 \ttbar measurement, we present the recent studies from the \ljets and \dilep final states.

 \subsection{Inclusive cross section measurement in the \texorpdfstring{\ljets}{ljets} final states}
 The measurement is performed in the \ejets and \mujets final states with 35.9\fbinv integrated
 luminosity~\cite{CMS-PAS-TOP-20-002}. An event-level discriminant based on Boosted Decision Trees 
 is used to extract the cross section. The events are divided into different signal and control 
 regions based on the number of jets and b-tagged jets. A simultaneous fit is performed on the
 discriminant combining 3 categories and 2 final states. The dominant source of systematic 
 uncertainty comes from the jet energy correction. The predicted (at NLO) and 
 measured cross sections are $79.5^{+1.9+2.0}_{-1.8-1.4}$ pb and 
 $89 \pm 4 (\text{stat}) \pm 12 (\text{syst})$ pb, respectively. They are in agreement within the
 uncertainties.

 \subsection{Differential cross section measurement in the \texorpdfstring{\dilep}{dilep} final states}
 The differential cross section is measured with 35.9\fbinv integrated luminosity in the \dilep final
 states with different lepton flavors (electron or muon) as a function of the six 
 variables~\cite{CMS-PAS-TOP-19-003}: \pt of the leading lepton, \pt of the jet, angular difference
 \deltaPhiVar, longitudinal momentum \pzvar, invariant mass \invmassvar, and transverse mass 
 \transmassvar. Signal extraction is performed by subtracting backgrounds from data. The jet energy 
 correction uncertainties are the dominant ones. Predicted and measured cross sections are in 
 agreement within the uncertainties across different bins of all variables.
