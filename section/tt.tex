%================
% tt
%================
Due to the higher cross section among all \PQt quark production processes, the \ttbar production process is extensively studied at the LHC.
Depending on the subsequent decays of the two \PQt quarks, the final states consist of either all
jets, \ljets, or \dilep. All jet final states have larger size of the selected data set but more 
multijet background whereas the \dilep have fewer statistics but small background. The \ljets 
final states fall in between the two. However, the measurement of cross sections from all these 
final states is needed in order to study various physics beyond the SM. In this proceedings, we 
present the measurement from the \ljets and \dilep final states for \ttbar production.


\subsection{Inclusive and differential cross section measurements in the \texorpdfstring{\ljets}{ljets} final states}
The measurement of the \ttbar cross section is performed using 137 \fbinv integrated luminosity in 
the \ejets and \mujets final states~\cite{CMS-PAS-TOP-20-001}. In order to increase the 
sensitivity, the events are further divided into boosted and resolved categories based on the 
kinematics of the decay products of the \ttbar pair. The final cross section is extracted by a 
simultaneous fit combining events from both final states and all categories. Various distributions 
such as the transverse momentum of the \PQt quark, invariant mass of \ttbar, etc are used to 
measure the differential and double differential cross-sections at parton and particle levels. A 
neural network is exploited in the reconstruction of variables from boosted \PQt quarks. 
A $\chi^2$ test is performed to compare the measurements with several predictions. The dominant 
source of systematic uncertainty comes from jet energy correction.

The measured value of the inclusive cross section, $791\pm 25$ pb, is in agreement with the corresponding 
predicted value of $832\pm 46$ pb. One of the most 
striking features of this measurement is that the measured cross section is more precise 
(3.2\% uncertainty) as compared to the predicted value (5.5\% uncertainty). The measured and 
predicted differential and double differential cross sections as functions of various variables are 
in agreement within the uncertainties for most of the variables. However, there is a slight 
discrepancy in the double differential measurement for higher \pt of the hadronically decaying \PQt
quark in the range $ 0 < \pt (\ttbar) < 120 \GeV$. A similar over-prediction is 
also observed from the ATLAS experiment~\cite{ATLAS:2020ccu} and signals a possible mismodeling 
of the Monte Carlo event generator used for the simulation of predicted events. 

\subsection{Inclusive cross section measurement in the \texorpdfstring{\dilep}{dilep} final states at 5.02 \texorpdfstring{\TeV}{TeV} center-of-mass energy}
As shown in Figure~\ref{fig:xss}, all measurements are performed at 7, 8, and 13 \TeV center-of-mass 
energies. This is the first measurement at 5.02 \TeV, which provides another test for the 
SM at lower energy~\cite{CMS-PAS-TOP-20-004}. The measurement is performed using 0.304\fbinv 
integrated luminosity in the \ejets and \mujets final states. The cross section is extracted 
by performing the fit on the total number of events after applying all selections. The dominant source 
of systematic uncertainty comes from the jet energy correction. The predicted cross section at the
next-to-leading order (NLO) in QCD and observed value are 
$66.8^{+1.9}_{-2.3}\text{(scale)}\pm 1.7\text{(PDF)}^{+1.4}_{-1.3}(\alpha_S)$ pb and 
$60.3 \pm 5.0 \text{(stat)} \pm 2.8 \text{(syst)} \pm 0.9 \text{(lumi)}$ pb, respectively. They
agree within the uncertainties. 

\subsection{Inclusive cross section measurement in the \texorpdfstring{\dilep}{dilep} final states including \texorpdfstring{\PGt}{tau} lepton}
This is the first measurement involving tau leptons~\cite{CMS:2019snc} and provides another 
way of checking lepton flavor universality. With the third generation of lepton and quarks,
it is sensitive to beyond SM contributions such as the production of charged Higgs boson. In 
this analysis, one of the leptons from \dilep final states is required to be a hadronically 
decaying \PGt and the other one is either an electron or muon. The measurement is performed at 
13 \TeV using 35.9\fbinv integrated luminosity. The cross section is extracted using the profile 
likelihood method based on the transverse mass of the lepton and missing transverse energy. The  
QCD multijet background is estimated from data. The main sources of systematic uncertainty are 
from $\tau_h$ identification and misidentification. The measured cross section combining both 
channels is  
$\sigma_\ttbar(\ell\tau_h) = 781 \pm 7 \text{(stat)} \pm 62 \text{(syst)} \pm 20 \text{(lumi)}$ pb, 
which is in agreement with the corresponding predicted value. The ratio of this with the same 
flavor cross section 
$R_{\ell\tau_h/\ell\ell} = 0.973 \pm 0.009 \text{(stat)} \pm 0.066 \text{(syst)}$ is close to 1 
within the uncertainties. Hence the lepton flavor universality violation is not observed.

