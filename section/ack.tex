 \section*{Acknowledgements}
 We congratulate our colleagues in the CERN accelerator departments for the excellent performance of the LHC
 and thank the technical and administrative staffs at CERN and at other CMS institutes for their contributions
 to the success of the CMS effort. In addition, we gratefully acknowledge the computing centres and personnel
 of the Worldwide LHC Computing Grid for delivering so effectively the computing infrastructure essential to
 our analyses. Finally, we acknowledge the enduring support for the construction and operation of the LHC and
 the CMS detector provided by the following funding agencies: BMBWF and FWF (Austria); FNRS and FWO (Belgium);
 CNPq, CAPES, FAPERJ, FAPERGS, and FAPESP (Brazil); MES (Bulgaria); CERN; CAS, MoST, and NSFC (China);
 COLCIENCIAS (Colombia); MSES and CSF (Croatia); RPF (Cyprus); SENESCYT (Ecuador); MoER, ERC IUT, PUT and ERDF
 (Estonia); Academy of Finland, MEC, and HIP (Finland); CEA and CNRS/IN2P3 (France); BMBF, DFG, and HGF
 (Germany); GSRT (Greece); NKFIA (Hungary); DAE and DST (India); IPM (Iran); SFI (Ireland); INFN (Italy); MSIP
 and NRF (Republic of Korea); MES (Latvia); LAS (Lithuania); MOE and UM (Malaysia); BUAP, CINVESTAV, CONACYT,
 LNS, SEP, and UASLP-FAI (Mexico); MOS (Montenegro); MBIE (New Zealand); PAEC (Pakistan); MSHE and NSC
 (Poland); FCT (Portugal); JINR (Dubna); MON, RosAtom, RAS, RFBR, and NRC KI (Russia); MESTD (Serbia); SEIDI,
 CPAN, PCTI, and FEDER (Spain); MOSTR (Sri Lanka); Swiss Funding Agencies (Switzerland); MST (Taipei);
 ThEPCenter, IPST, STAR, and NSTDA (Thailand); TUBITAK and TAEK (Turkey); NASU and SFFR (Ukraine); STFC (United
 Kingdom); DOE and NSF (USA).
